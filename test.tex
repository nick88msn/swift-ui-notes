\PassOptionsToPackage{unicode=true}{hyperref} % options for packages loaded elsewhere
\PassOptionsToPackage{hyphens}{url}
%
\documentclass[]{article}
\usepackage{lmodern}
\usepackage{amssymb,amsmath}
\usepackage{ifxetex,ifluatex}
\usepackage{fixltx2e} % provides \textsubscript
\ifnum 0\ifxetex 1\fi\ifluatex 1\fi=0 % if pdftex
  \usepackage[T1]{fontenc}
  \usepackage[utf8]{inputenc}
  \usepackage{textcomp} % provides euro and other symbols
\else % if luatex or xelatex
  \usepackage{unicode-math}
  \defaultfontfeatures{Ligatures=TeX,Scale=MatchLowercase}
\fi
% use upquote if available, for straight quotes in verbatim environments
\IfFileExists{upquote.sty}{\usepackage{upquote}}{}
% use microtype if available
\IfFileExists{microtype.sty}{%
\usepackage[]{microtype}
\UseMicrotypeSet[protrusion]{basicmath} % disable protrusion for tt fonts
}{}
\IfFileExists{parskip.sty}{%
\usepackage{parskip}
}{% else
\setlength{\parindent}{0pt}
\setlength{\parskip}{6pt plus 2pt minus 1pt}
}
\usepackage{hyperref}
\hypersetup{
            pdfborder={0 0 0},
            breaklinks=true}
\urlstyle{same}  % don't use monospace font for urls
\setlength{\emergencystretch}{3em}  % prevent overfull lines
\providecommand{\tightlist}{%
  \setlength{\itemsep}{0pt}\setlength{\parskip}{0pt}}
\setcounter{secnumdepth}{0}
% Redefines (sub)paragraphs to behave more like sections
\ifx\paragraph\undefined\else
\let\oldparagraph\paragraph
\renewcommand{\paragraph}[1]{\oldparagraph{#1}\mbox{}}
\fi
\ifx\subparagraph\undefined\else
\let\oldsubparagraph\subparagraph
\renewcommand{\subparagraph}[1]{\oldsubparagraph{#1}\mbox{}}
\fi

% set default figure placement to htbp
\makeatletter
\def\fps@figure{htbp}
\makeatother


\date{}

\begin{document}

\hypertarget{swift-ui}{%
\section{Swift UI}\label{swift-ui}}

A declarative user interface for developing apps on every Apple
platform. SwiftUI provides views, controls, and layout structures for
declaring your app user interface. The framework provides event handlers
for delivering taps, gestures, and other types of input, and tools to
manage the flow of data from app's models down to the views and
controls.
\href{https://developer.apple.com/documentation/swiftui/}{Apple Official
Documnentation}

Swift UI uses the Swift programming language
\href{https://github.com/nick88msn/swift-starter-kit}{Swift Notes}

\hypertarget{your-first-view}{%
\subsection{Your first view}\label{your-first-view}}

Swift UI does a great job separating the interface from the logic. A
View and most of the types we are gonna use in Swift UI are Struct.
Struct are commons in other language programs are an abbreviation for
Data Structure. A struct can be seen as a collections of variables,
however we'll see they are powerful types on which most of the Swift UI
elements are built up. Struct have variables but can also have
functions. They are similar to classes but we'll see they lack
inheritance. Struct are not object oriented things. Swift support also
classes and supports both Object Oriented Programming or Functional
Programming. We'll use OOP to hook our models to the UI, while our logic
and UI will basically a set of structs.

A basic Swift UI view can be:

\begin{verbatim}
struct MyFirstView: View {
    var body: some View {
        Text("Hello World")
    }

}
\end{verbatim}

In functional programming the behavior of things is important. In this
case the struct MyFirstView ``behaves'' like a View. Behaving like
something in functional programming usually mean we inherit some work
already done for us but also means we may need to provide something to
conform to the type. In the case of View, we'll see that View is a Swift
protocol. It is possible to create custom views by declaring types that
conform to the
\href{https://developer.apple.com/documentation/swiftui/view}{View}
protocol. To conform to the protocol we need to implement the required
body computed property to provide the content for our custom view.

\ldots{}..

\hypertarget{animation}{%
\section{Animation}\label{animation}}

Animation is very important in a mobile UI. Swift makes very easy to do.
There are basically two ways to do animation: 1. by animating a Shape 2.
by animating Views via their ViewModifiers

\hypertarget{so-what-is-a-viewmodifier}{%
\subsection{So what is a
ViewModifier?}\label{so-what-is-a-viewmodifier}}

View modifiers are all those little functions that modified our Views
(like aspectRatio, padding etc) They are (likely) turning right aroud
and calling a function in View called Modifier. e.g.~.aspectRatio(2/3)
is likely something like .modifier(AspectModifier(2/3)) AspectModifier
can be anything that conforms to the ViewModifier protocol.

\href{https://developer.apple.com/documentation/swiftui/viewmodifier}{Apple
Documentation - View Modifier} A modifier that you apply to a view or
another view modifier, producing a different version of the original
value.

ViewModifier is a protocol that lets us create a reusable modifier that
can be applied to any view.

Conceptually, this protocol is sort of like this\ldots{}

\begin{verbatim}
protocol ViewModifier{
    typealias Content //the type of the View passed to body(content:)
    func body(content: Content) -> some View {
        //return some View that almost certainly contains the View content
    }
}
\end{verbatim}

An example of view modifier is:

\begin{verbatim}

struct BorderdLabel: ViewModifier{
    var isSet: Bool
    func body(content: Content) -> some View {
        content
            .font(.caption2)
            .padding(10)
            .overlay(RoundedRectangle(cornerRadius:10))
    }.foregroundColor(Color.blue)
}
\end{verbatim}

Where Content is the View that we are actually going to modify. The code
is similar to the View code, but we have func body(content:) instead of
var body. That's beacause ViewModifiers are Views, and writing the code
for one is almost identical. There is also a special ViewModifier,
GeometryEffect, for building geometry modifiers (scaling, rotation
etc..).

\hypertarget{how-to-apply-a-viewmodifier}{%
\subsection{How to apply a
ViewModifier?}\label{how-to-apply-a-viewmodifier}}

We have two ways to apply a view modifier: 1. Apply the modifier
directly to the view (e.g.~Text(``Some
Text'').modifier(BorderedLabel())) 2. Create an extension that use a
modifier to the View itself and return the View modified

\begin{verbatim}
extension View{
    func borderedLabel(isSet: Bool) -> some View {
        return self.modifier(BorderedLabel(isSet: isSet))
    }
}
\end{verbatim}

It is important to highlight a couple of things on how this works. 1.
The content argument is just the Text view we pass 2. We pass the
arguments like in the View 3. We could have an init (it is allowed)

These arguments are crucial in ViewModifiers since when this changes it
will kick off an animation. ViewModifiers always return a View (some
View) not multiple views or something else. They take a view and return
a new one (remember they are structs, they are read only in memory so we
return always a new one).

P.S. (ViewModifiers do not have var body so you do not need the SwiftUI
Template when implementing on a single new file)

\hypertarget{animation-1}{%
\subsection{\texorpdfstring{\href{https://www.youtube.com/watch?v=PoeaUMGAx6c}{Animation}}{Animation}}\label{animation-1}}

Basics: - Important takeaways about Animation - Only changes can be
animated. Changes to what? - arguments to ViewModifiers\\
- arguments to the cretion of Shapes - the existence (or not) of a View
in the UI - Animation is showing the user changes that have already
happened (i.e.the recent past) - Our code does something, makes a
change, and only then swift ui triggers the animation - ViewModifiers
are the primary change agents in the UI - It is important to understand
that: - A change to a ViewModifier's arguments has to happen after the
View is initially put in the UI - In other words: only changes in a
ViewModifier's arguments since it joined the UI are animated. - Not all
ViewModifier's arguments are animatable (e.g.~.font is not), but most
are. - When a View arrives or departs, the entire thing is animated as a
unit: - A View coming on-screen is only animated if it is joining a
container that is already in the UI. - A View going off-screen is only
animated if it is leaving a container that is staying in the UI. -
ForEach and if-else in ViewBuilders are common ways to make VIews come
and go

How do we make an animation ``go''? 1. Implicitly (automatically), by
using the view modifier .animation(Animation) 2. Explicitly, by wrapping
withAnimation(Animation) \{ \} around code that might change things 3.
Indipendently, By making Views be included or excluded from the UI

Again, all of the above only cause animations to ``go'' if the View is
already part of the UI (or if the View is joining a container that is
already part of the UI)

\hypertarget{implicit-animation}{%
\subsubsection{Implicit Animation}\label{implicit-animation}}

\begin{itemize}
\tightlist
\item
  Automatic animation, essentially marks a View so that \ldots{}
\item
  All ViewModifier arguments that precede the animation modifier will
  always be animated
\item
  the changes are animated with the duration and ``curve'' you specify
\end{itemize}

To create a simple implicit animation add a .animation(Animation) view
modifier to the View you want to auto-animate.

\begin{verbatim}
Text("👻")
    .opacity(scary ? 1 : 0)
    .rotationEffect(Angle.degrees(upsideDown ? 180 : 0))
    .animation(Animation.easeInOut)
\end{verbatim}

Now whenever scary or upsideDown changes, the opacity/rotation will be
animated. All changes to arguments to animatable view modifiers
preceding .animation are animated. If we put something after
.animation() those will not be animated (since animation is a view
modifier) Without .animation(), the changes to opacity/rotation would
appear instantly on screen.

Warning! The .animation modifier does not work well on a container. A
container just propagates the .animation modifier to all the Views it
contains. In other words, .animation does not work not like .padding, it
works more like .font. It is good for single view like Text, Image etc.

\begin{verbatim}
//
//  ImplicitAnimation.swift
//  Memorize
//
//  Created by nick88msn on 08/06/21.
//

import SwiftUI

struct AnimationView: View {
    @State var isScary: Bool = true
    @State var isUpsideDown: Bool = false
    
    var body: some View {
        VStack{
            Spacer()
            Text("👻")
                .font(.largeTitle)
                .opacity(isScary ? 1 : 0)
                .rotationEffect(Angle.degrees(isUpsideDown ? 180 : 0))
                .animation(.easeInOut)
            Button(action: {
                isScary.toggle()
                isUpsideDown.toggle()
            }, label: {
                Text(isScary ? "Press me" : "Bring me back")
            })
            .padding()
            .font(.title)
            .foregroundColor(isScary ? .primary : .secondary)
            Spacer()
        }
    }
}


struct AnimationView_Previews: PreviewProvider {
    static var previews: some View {
        return AnimationView()
            .previewDevice("iPhone 12 Pro")
            .preferredColorScheme(.dark)
    }
}
\end{verbatim}

The argument of the .animation() ViewModifier is an Animation struct. It
lets you control things about an animation such as: - its duration - its
delay - whether the animation repeat a bunch of times or even loops
endlessly in repeatForever - its curve to control the rate at which the
animation plays out: - .linear, with a consistent rate throughout -
.easeInOut, starts out the animation slowly, picks up speed, then slows
at the end - .spring, provides soft landing (bounce) for the end of the
animation

\hypertarget{implicit-vs-explicit-animation}{%
\subsubsection{Implicit vs Explicit
Animation}\label{implicit-vs-explicit-animation}}

Implicit animation are usually not the primary source of animation
behavior. They are mostly used on ``leaf'' (i.e.~non container like
Text, Image etc) Views. Or, more generally, on Views that are typically
working independently of other Views.

A more common cause of animation is a change in our Model. Or more
generally, changes in response to some user action. For these changes,
we want a whole bunch of Views to animate together. For that, we use
Explicit Animation.

\hypertarget{explicit-animation}{%
\subsubsection{Explicit Animation}\label{explicit-animation}}

Explicit Animation create an animation transaction during which all
eligible changes mase as a result of executing a block of code will be
animated together. It is very easy to create one, you supply the
Animation (duration, curve etc) to use and the block of code.

\begin{verbatim}
withAnimation(.linear(duration:2)){
    // do something that will cause ViewModifier/Shape arguments to change somewhere
}
\end{verbatim}

Explicit animations are almost always wrapped around calls to ViewModel
Intent functions. But they are also wrapped around things that only
change then UI like ``entering editing mode.''. It is fairly rare for
code that handles a user gesture to not be wrapped in a withAnimation.

Explicit Animations do not override an implicit animation. Those
implicit animation are kind of indipendent and are not affected by an
explicit animation.

\hypertarget{transitions}{%
\subsection{Transitions}\label{transitions}}

Transitions specify how to animate the arrival/departure of Views. It
only works for Views that are inside Containers that are already
On-Screen. Under the covers, a transition is nothing more than a pair of
ViewModifiers, One of the modifiers is the ``before'' modification of
the View that's on the move. The other modifier is the ``after''
modification of the View on the move. Thus a transition is just a
version of a ``changes in arguments to ViewModifiers'' animation.

It is possible to have an asymmetric transition (two different
transition for in and out). An asym transition has 2 pairs of
ViewModifiers. One pair for when the View appears (insertion), and
another pair for when the View disappers (removal).

Example: A view fades in when it appears, but then flies across the
screen when it disappears. Mostly we use "pre-canned' transitions
(opacity, scaling, moving across the screen). They are static var/funcs
on the AnyTransition struct.

All the Transition API is `type erased'. We use the struct AnyTransition
which erases type info for the underlying ViewModifiers. This makes a
lot easier to work with transitions.

Some of the built-in transitions: - AnyTransition.opacity (uses .opacity
modifier to fate the View in and out) - AnyTransition.scale (uses .frame
modifier to expand/shring the View as it comes and goes) -
AnyTransition.offset(CGSize) (use .offset modifier to move the View as
it comes and goes) - AnyTransition.modifier(active:identity:) (you
provide the two ViewModifiers to use)

\hypertarget{how-to-apply-transitions}{%
\subsubsection{How to apply
transitions}\label{how-to-apply-transitions}}

How do we specify which kind of transition to use when a View
arrive/departs? Using .transition(). Example using two built-in
transitions, .scale and .identity:

\begin{verbatim}
ZStack {
    if isFacedUp{
        RoundedRectangle(cornerRadius:10).stroke()
        Text("👻").transition(AnyTransition.scale)
    } else {
        RoundedRectangle(cornerRadius: 10).transition(AnyTransition.identity)
    }
}
\end{verbatim}

If isFacedUp changed from false to true (and ZStack was already on
screen and we were explicitly animating) the back would disappear
instantly, Text would grow in from nothing, front RondedRectangle would
fade in.

Unlike .animation(), .transition() does not get redistributed to a
container's content Views. So putting .transition() on the ZStack above
only works if the entire ZStack came/went (Group and ForEach do
distribute .transition() to their content Views, however).

.transition() is just specifying what the ViewModifiers are. It does not
cause any animation to occurr. In other words, think of the word
transition as a noun here, not a verb. You are declaring what transition
to use, not causing the transition to occur. Most probably you are going
to trigger the transition with an explicit animation.

\hypertarget{setting-animation-details-for-a-transition}{%
\subsubsection{Setting Animation Details for a
Transition}\label{setting-animation-details-for-a-transition}}

You can set an animation (curve/duration/etc.) to use for a transition.
AnyTransition structs have a .animation(Animation) of their own you can
call. This sets the Animation parameters to use to animate the
transition.
e.g.~.transition(AnyTransition.opacity.animation(.linear(duration: 20)))

\hypertarget{matched-geometry-effect}{%
\subsection{Matched Geometry Effect}\label{matched-geometry-effect}}

Sometimes you want a View to move from one place on the screen to
another (and possibly resize along the way). If the View is moving to a
new place in its same container, this is no problem. ``Moving'' like
this is just animating the .position ViewModifier's (.position is what
HStack, LazyVGrid, etc., use to position the Views inside them). This
kind of thing happens automatically when you explicitly animate.

But what if the View is ``moving'' from one container to a different
container? This is not really possible. Instead, you need a View in the
``Source'' position and a different one in the ``destination'' position.
And then, you must ``match'' their geometries up as one leaves the UI
and the other arriver. So this is similat to .transition in that it is
animating Views' coming and going in the UI. It's just that it's
particular to the case where a pair of Views' arrivals/departures are
synced.

In the memorize game a great example of this would be ``dealing cards
off of a deck''. The ``deck'' might well be its own View off to the
side. WHen a card is ``dealt'' from the deck, it needs to fly from there
to the game. But the deck and game's main View are not in the same
LazyVGrid or anything. How do we handle this?

\end{document}
